%%%%%%%%%%%%% Types d'exercices

\newcommand{\AD}[2]{
\ifthenelse{\equal{#1}{1}}
{%condition2 vraie
\vspace{0.4cm}
\stepcounter{cptex}
\tikz\node[rounded corners=0pt,draw,fill=bleu2]{\color{white}\textbf{ \thecptex}}; \quad  {\color{bleu2}\textbf{ Application directe}}
\input{#2}
}% fin condition2 vraie
{%condition2 fausse
\stepcounter{cptex}
\vspace{0.4cm}
\tikz\node[rounded corners=0pt,draw,fill=bleu2]{\color{white}\textbf{ \thecptex}}; \quad  {\color{bleu2}\textbf{ Application directe}} En temps libre
}% fin condition2 fausse
} % fin de la procédure

\newcommand{\TP}[2]{
\ifthenelse{\equal{#1}{1}}
{%condition2 vraie
\vspace{0.4cm}
\stepcounter{cptex}
\tikz\node[rounded corners=0pt,draw,fill=eduscol4P]{\color{white}\textbf{ \thecptex}}; \quad  {\color{eduscol4P}\textbf{ Travail personnel}}
\input{#2}
}% fin condition2 vraie
{%condition2 fausse
\stepcounter{cptex}
\vspace{0.4cm}
\tikz\node[rounded corners=0pt,draw,fill=eduscol4P]{\color{white}\textbf{ \thecptex}}; \quad  {\color{eduscol4P}\textbf{ Travail personnel}} En temps libre
}% fin condition2 fausse
} % fin de la procédure

 
\newcommand{\Fl}[2]{
\ifthenelse{\equal{#1}{1}}
{%condition2 vraie
\vspace{0.4cm}
\stepcounter{cptex}
\tikz\node[rounded corners=2pt,draw,fill=red]{\color{white}\textbf{ \thecptex}}; \quad  {\color{red}\textbf{ Question flash}}
\input{#2}
}% fin condition2 vraie
{%condition2 fausse
\stepcounter{cptex}
\vspace{0.4cm}
\tikz\node[rounded corners=2pt,draw,fill=red]{\color{white}\textbf{ \thecptex}}; \quad  {\color{red}\textbf{ Question flash}} dans le cahier d'exercices
}% fin condition2 fausse
} % fin de la procédure


\newcommand{\ExD}[1]{ % Exercice directement dans la page
\vspace{0.4cm}
\stepcounter{cptex}
\tikz\node[rounded corners=0pt,draw,fill=bleu2]{\color{white}\textbf{ \thecptex}}; \quad  {\color{bleu2}\textbf{ Exercice}}
\ \\
}



\newcommand{\AVDM}[1]{ %  directemet dans la page
\vspace{0.4cm}
\stepcounter{cptavdm}
\tikz\node[rounded corners=0pt,draw,fill=purple]{\color{white}\textbf{ \thecptavdm}}; \quad  {\color{purple}\textbf{A vous de manipuler}}
\ \\
}


\newcommand{\act}[1]{ %  directemet dans la page
\vspace{0.4cm}
\stepcounter{cptavdm}
\tikz\node[rounded corners=0pt,draw,fill=purple]{\color{white}\textbf{ Activité \thecptavdm}};  
 
\ \\
}

% Ligne de commande
\newcommand{\cmd}[1]{\fcolorbox{gray!40}{gray!10}{ \vspace{0.1cm} \texttt{ #1 }  \vspace{0.1cm}}}
 
\newcommand{\Tab}[2]{
\ifthenelse{\equal{#1}{1}}
{%condition2 vraie
\vspace{0.4cm}
\stepcounter{cptex}
\tikz\node[rounded corners=0pt,draw,fill=gray]{\color{white}\textbf{ \thecptex}}; \quad  {\color{gray}\textbf{ Exercice Tableur}}
\input{#2}
}% fin condition2 vraie
{%condition2 fausse
\stepcounter{cptex}
\vspace{0.4cm}
\tikz\node[rounded corners=0pt,draw,fill=gray]{\color{white}\textbf{ \thecptex}}; \quad  {\color{gray}\textbf{ Exercice Tableur}} En temps libre
}% fin condition2 fausse
} % fin de la procédure


\newcommand{\Exo}[2]{
\ifthenelse{\equal{#1}{1}}
{%condition2 vraie
\vspace{0.4cm}
\stepcounter{cptex}
\tikz\node[rounded corners=0pt,draw,fill=bleu2]{\color{white}\textbf{ \thecptex}}; \quad  {\color{bleu2}\textbf{ Exercice d'application}}
\input{#2}
}% fin condition2 vraie
{%condition2 fausse
\vspace{0.4cm}
\stepcounter{cptex}
\tikz\node[rounded corners=0pt,draw,fill=bleu2]{\color{white}\textbf{ \thecptex}}; \quad  {\color{bleu2}\textbf{ Exercice d'application}} En temps libre
}% fin condition2 fausse
} % fin de la procédure



\newcommand{\Alg}[2]{
\ifthenelse{\equal{#1}{1}}
{%condition2 vraie
\vspace{0.4cm}
\stepcounter{cptex}
\tikz\node[rounded corners=0pt,draw,fill=gray]{\color{white}\textbf{ \thecptex}}; \quad  {\color{gray}\textbf{ Algorithmique}}
\input{#2}
}% fin condition2 vraie
{%condition2 fausse
\vspace{0.4cm}
\stepcounter{cptex}
\tikz\node[rounded corners=0pt,draw,fill=gray]{\color{white}\textbf{ \thecptex}}; \quad  {\color{gray}\textbf{ Algorithmique}} En temps libre
}% fin condition2 fausse
} % fin de la procédure


\newcommand{\EPC}[3]{ % Exerice par compétence de niveau 1
\ifthenelse{\equal{#1}{1}}
{%condition2 vraie
\vspace{0.4cm}
\stepcounter{cptex}
\tikz\node[rounded corners=0pt,draw,fill=bleu2]{\color{white}\textbf{ \thecptex}}; \quad  {\color{bleu2}\textbf{#3}}
\input{#2}
}% fin condition2 vraie
{%condition2 fausse
\vspace{0.4cm}
\stepcounter{cptex}
\tikz\node[rounded corners=2pt,draw,fill=eduscol4P]{\color{white}\textbf{ \thecptex}}; \quad  {\color{eduscol4P} \textbf{En temps libre.} \textbf{ #3}} 
\input{#2}
}% fin condition2 fausse
} % fin de la procédure



\newcommand{\EPCcor}[3]{ % Exerice par compétence de niveau 1
\vspace{0.4cm}
\tikz\node[rounded corners=2pt,draw,fill=eduscol4P]{\color{white}\textbf{ #1}}; \quad  {\color{eduscol4P} \textbf{En temps libre.} \textbf{ #3}} 
\input{#2}
} % fin de la procédure 
 



\newcommand{\EPCB}[3]{ % Exerice par compétence de niveau 2
\ifthenelse{\equal{#1}{1}}
{%condition2 vraie
\vspace{0.4cm}
\stepcounter{cptex}
\tikz\node[rounded corners=0pt,draw,fill=bleu1]{\color{white}\textbf{ \thecptex}}; \quad  {\color{bleu1}\textbf{#3}}
\input{#2}
}% fin condition2 vraie
{%condition2 fausse
\vspace{0.4cm}
\stepcounter{cptex}
\tikz\node[rounded corners=0pt,draw,fill=bleu1]{\color{white}\textbf{ \thecptex}}; \quad  {\color{bleu1}\textbf{ #3}} d'exercices
}% fin condition2 fausse
} % fin de la procédure



\newcommand{\EPCC}[3]{% % Exerice par compétence de niveau 3
\ifthenelse{\equal{#1}{1}}
{%condition2 vraie
\vspace{0.4cm}
\stepcounter{cptex}
\tikz\node[rounded corners=0pt,draw,fill=black]{\color{white}\textbf{ \thecptex}}; \quad  {\color{black}\textbf{#3}}
\input{#2}
}% fin condition2 vraie
{%condition2 fausse
\vspace{0.4cm}
\stepcounter{cptex}
\tikz\node[rounded corners=0pt,draw,fill=black]{\color{white}\textbf{ \thecptex}}; \quad  {\color{black}\textbf{ #3}} d'exercices
}% fin condition2 fausse
} % fin de la procédure


\newcommand{\EPCP}[3]{% % Exerice par compétence avec  Python
\ifthenelse{\equal{#1}{1}}
{%condition2 vraie
\vspace{0.4cm}
\stepcounter{cptex}
\tikz\node[rounded corners=0pt,draw,fill=bleu2]{\color{white}\textbf{ \thecptex}};  \tikz\node[rounded corners=2pt,draw,fill=yellow]{\color{bleu2}\textbf{Python}} ;\quad  {\color{bleu2} \textbf{#3}}
\input{#2}
}% fin condition2 vraie
{%condition2 fausse
\vspace{0.4cm}
\stepcounter{cptex}
\tikz\node[rounded corners=0pt,draw,fill=black]{\color{white}\textbf{ \thecptex \quad  Python}}; \quad  {\color{black}\textbf{ #3}} d'exercices
}% fin condition2 fausse
} % fin de la procédure


\newcommand{\EPCG}[3]{% % Exerice par compétence avec  Python
\ifthenelse{\equal{#1}{1}}
{%condition2 vraie
\vspace{0.4cm}
\stepcounter{cptex}
\tikz\node[rounded corners=0pt,draw,fill=bleu2]{\color{white}\textbf{ \thecptex}}; \tikz\node[rounded corners=2pt,draw,fill=violet]{\color{white}\textbf{Geogebra}};\quad  {\color{violet} \textbf{#3}}
\input{#2}
}% fin condition2 vraie
{%condition2 fausse
\vspace{0.4cm}
\stepcounter{cptex}
\tikz\node[rounded corners=0pt,draw,fill=violet]{\color{white}\textbf{ \thecptex \quad Geogebra}}; \quad  {\color{violet}\textbf{ #3}} d'exercices
}% fin condition2 fausse
} % fin de la procédure

\newcommand{\EPCN}[1]{ 
\vspace{0.4cm}
\stepcounter{cptex}
\tikz\node[rounded corners=0pt,draw,fill=bleu2]{\color{white}\textbf{\thecptex}}; \quad  {\color{bleu2}\textbf{ #1}}
}


\newcommand{\EPCNM}[1]{ 
\vspace{0.4cm}
\stepcounter{cptex}
\tikz\node[rounded corners=0pt,draw,fill=gray]{\color{white}\textbf{\thecptex}}; \quad  {\color{gray}\textbf{ #1}}
}


\newcommand{\EPCNA}[1]{ 
\vspace{0.4cm}
\stepcounter{cptex}
\tikz\node[rounded corners=0pt,draw,fill=black]{\color{white}\textbf{\thecptex}}; \quad  {\color{black}\textbf{ #1}}
}

\newcommand{\EPCNP}[1]{ 
\vspace{0.4cm}
\stepcounter{cptex}
\tikz\node[rounded corners=0pt,draw,fill=yellow]{\color{bleu2}\textbf{\thecptex} Python}; \quad  {\color{bleu2}\textbf{ #1}}
}

\newcommand{\Auto}[1]{ 
\vspace{0.4cm}
\stepcounter{cptex}
\tikz\node[rounded corners=0pt,draw,fill=gray]{\color{white}\textbf{\thecptex}}; \quad  {\color{gray}\textbf{ #1}}
}



\newenvironment{ExC}[1][]{%
\medskip \begin{tcolorbox}[widget,colback=bleu1!15,colframe=bleu2!75!white,
adjusted title= Exercice corrigé.  {#1}     ]}
{%
\end{tcolorbox}\par}



\newcommand{\FlC}[3]{
\ifthenelse{\equal{#1}{1}}
{%condition2 vraie
\vspace{0.4cm}
\stepcounter{cptex}
\tikz\node[rounded corners=2pt,draw,fill=red]{\color{white}\textbf{ \thecptex}}; \quad  {\color{red}\textbf{ Question flash} #3}
\input{#2}
}% fin condition2 vraie
{%condition2 fausse
\stepcounter{cptex}
\vspace{0.4cm}
\tikz\node[rounded corners=2pt,draw,fill=red]{\color{white}\textbf{ \thecptex}}; \quad  {\color{red}\textbf{ Question flash} #3} En temps libre
}% fin condition2 fausse
} % fin de la procédure




\newcommand{\Dec}[2]{
\ifthenelse{\equal{#1}{1}}
{%condition2 vraie
\vspace{0.4cm}
\stepcounter{cptex}
\tikz\node[rounded corners=0pt,draw,fill=bleu2]{\color{white}\textbf{ \thecptex}}; \quad  {\color{bleu2}\textbf{ Exercice de découverte}}
\input{#2}
}% fin condition2 vraie
{%condition2 fausse
\stepcounter{cptex}
\vspace{0.4cm}
\tikz\node[rounded corners=0pt,draw,fill=bleu2]{\color{white}\textbf{ \thecptex}}; \quad  {\color{bleu2}\textbf{ Exercice de découverte}} En temps libre
}% fin condition2 fausse
} % fin de la procédure


\newcommand{\PO}[2]{
\ifthenelse{\equal{#1}{1}}
{%condition2 vraie
\vspace{0.4cm}
\stepcounter{cptex}
\tikz\node[rounded corners=0pt,draw,fill=black]{\color{white}\textbf{ \thecptex}}; \quad  {\color{black}\textbf{ Défi}}
\input{#2}
}% fin condition2 vraie
{%condition2 fausse
\stepcounter{cptex}
\vspace{0.4cm}
\tikz\node[rounded corners=0pt,draw,fill=black]{\color{white}\textbf{ \thecptex}}; \quad  {\color{black}\textbf{ Défi}} En temps libre
}% fin condition2 fausse
} % fin de la procédure



\newcommand{\Exe}{
\vspace{0.4cm}
\stepcounter{cptex}
\tikz\node[rounded corners=0pt,draw,fill=bleu2]{\color{white}\textbf{ Exercice \thecptex}}; 
\\
}

\newcommand{\Dnb}{
\vspace{0.4cm}
\stepcounter{cptex}
\tikz\node[rounded corners=0pt,draw,fill=red]{\color{white}\textbf{ Exercice vu au DNB \thecptex}}; 
\\
}

\newcommand{\DnbDetails}[2]{
\vspace{0.4cm}
\tikz\node[rounded corners=0pt,draw,fill=red]{\color{white}\textbf{ Vu au DNB - #1 - Exercice #2  }}; 
\\
}





\newcommand{\ExeN}[1]{
\vspace{0.4cm}
\stepcounter{cptex}
\tikz\node[rounded corners=0pt,draw,fill=bleu2]{\color{white}\textbf{ Exercice \thecptex}}; 
\ifthenelse{#1 < 2} { \hfill{1 point}}{\hfill{#1 points}}
\\
}


\newcommand{\ExeSC}{
\vspace{0.4cm}
\tikz\node[rounded corners=0pt,draw,fill=bleu2]{\color{white}\textbf{ Exercice}}; 
\\
}


\newcommand{\Defi}{
\vspace{0.4cm}
\stepcounter{cptex}
\tikz\node[rounded corners=0pt,draw,fill=black]{\color{white}\textbf{ Approfondissement \thecptex}}; 
\\
}

\newcommand{\MP}{
\vspace{0.4cm}
\tikz\node[rounded corners=0pt,draw,fill=bleu2]{\color{white}\textbf{ Mini Projet possible}}; 
\\
}


\newcommand{\BAC}[3]{ %Vu au Bac avec le lieu et l'année et dissociation élève ou prof
\ifthenelse{\equal{#1}{1}}
{%condition2 vraie
\vspace{0.4cm}
\stepcounter{cptex}
\tikz\node[rounded corners=0pt,draw,fill=bleu3]{\color{white}\textbf{ \thecptex}}; 
\tikz\node[rounded corners=0pt,draw,fill=purple]{\color{white}\textbf{ Vu au Bac. #2 }};

\input{#3}
}% fin condition2 vraie
{%condition2 fausse
\vspace{0.4cm}
\stepcounter{cptex}
\tikz\node[rounded corners=0pt,draw,fill=purple]{\color{white}\textbf{ \thecptex}}; 
\quad  {\color{purple}\textbf{  Vu au baccalauréat #2}} En temps libre
}% fin condition2 fausse
} % fin de la procédure


\newcommand{\BACL}[1]{ %Vu au Bac avec le lieu et l'année
\vspace{0.4cm}
\stepcounter{cptex}
\tikz\node[rounded corners=0pt,draw,fill=bleu2]{\color{white}\textbf{ Exercice \thecptex}} ;
\tikz\node[rounded corners=0pt,draw,fill=red]{\color{white}\textbf{ Vu au Bac #1}}
;
\\
}



\newcommand{\BACS}{ %Vu au Bac sans le lieu et l'année
\vspace{0.4cm}
\stepcounter{cptex}
\tikz\node[rounded corners=0pt,draw,fill=bleu2]{\color{white}\textbf{ Exercice \thecptex}} ;
\tikz\node[rounded corners=0pt,draw,fill=red]{\color{white}\textbf{ Vu au Bac}}
;
\\
}


\newcommand{\ExP}{
\vspace{0.4cm}
\stepcounter{cptex}
\tikz\node[rounded corners=0pt,draw,fill=bleu4]{\color{white}\textbf{ Exercice \thecptex}}; 
\\
}
\newcommand{\ExF}{
\vspace{0.4cm}
\stepcounter{cptex}
\tikz\node[rounded corners=0pt,draw,fill=bleu1]{\color{white}\textbf{ Exercice \thecptex}}; 
\\
}

\newcommand{\PAPL}{
\vspace{0.4cm}
\stepcounter{cptex}
\tikz\node[rounded corners=0pt,draw,fill=red]{\color{white}\textbf{ Exercice \thecptex}}; \quad  {\color{red}\textbf{ Pour aller plus loin }}
\\
}

\newcommand{\Ast}{
\vspace{0.4cm}
\tikz\node[rounded corners=0pt,draw,fill=yellow]{\color{red}\textbf{ Astuce }}; 
\\
}

\newcommand{\Rec}[2]{
\ifthenelse{\equal{#1}{1}}
{%condition2 vraie
\vspace{0.4cm}
\stepcounter{cptex}
\tikz\node[rounded corners=0pt,draw,fill=bleu3]{\color{white}\textbf{ \thecptex}}; \quad  {\color{bleu3}\textbf{ Situation de recherche}}
\input{#2}
}% fin condition2 vraie
{%condition2 fausse
\stepcounter{cptex}
\vspace{0.4cm}
\tikz\node[rounded corners=0pt,draw,fill=bleu3]{\color{white}\textbf{ \thecptex}}; \quad  {\color{bleu3}\textbf{ Situation de recherche}} En temps libre
}% fin condition2 fausse
} % fin de la procédure



\newcommand{\App}[2]{
\ifthenelse{\equal{#1}{1}}
{%condition2 vraie
\vspace{0.4cm}
\stepcounter{cptex}
\tikz\node[rounded corners=0pt,draw,fill=bleu3]{\color{white}\textbf{ \thecptex}}; \quad  {\color{bleu3}\textbf{ Approfondissement}}
\input{#2}
}% fin condition2 vraie
{%condition2 fausse
\stepcounter{cptex}
\vspace{0.4cm}
\tikz\node[rounded corners=0pt,draw,fill=bleu3]{\color{white}\textbf{ \thecptex}}; \quad  {\color{bleu3}\textbf{ Approfondissement}} En temps libre
}% fin condition2 fausse
} % fin de la procédure


\newcommand{\Scr}[2]{
\ifthenelse{\equal{#1}{1}}
{%condition2 vraie
\vspace{0.4cm}
\stepcounter{cptex}
\tikz\node[rounded corners=0pt,draw,fill=orange]{\color{white}\textbf{ \thecptex}}; \quad  {\color{orange}\textbf{  Activité Scratch}}
\input{#2}
}% fin condition2 vraie
{%condition2 fausse
\stepcounter{cptex}
\vspace{0.4cm}
\tikz\node[rounded corners=0pt,draw,fill=orange]{\color{white}\textbf{ \thecptex}}; \quad  {\color{orange}\textbf{  Activité Scratch}} En temps libre
}% fin condition2 fausse
} % fin de la procédure


\newcommand{\Pyt}[2]{
\ifthenelse{\equal{#1}{1}}
{%condition2 vraie
\vspace{0.4cm}
\stepcounter{cptex}
\tikz\node[rounded corners=0pt,draw,fill=yellow]{\color{blue}\textbf{ \thecptex}}; \quad  {\color{blue}\textbf{   Python}}
\input{#2}
}% fin condition2 vraie
{%condition2 fausse
\vspace{0.4cm}
\stepcounter{cptex}
\tikz\node[rounded corners=0pt,draw,fill=yellow]{\color{blue}\textbf{ \thecptex}}; \quad  {\color{blue}\textbf{   Python}} En temps libre
}% fin condition2 fausse
} % fin de la procédure


\newcommand{\Prep}[3]{
\ifthenelse{\equal{#1}{1}}
{%condition2 vraie
\vspace{0.4cm}
\stepcounter{cptex}
\tikz\node[rounded corners=0pt,draw,fill=red]{\color{white}\textbf{ \thecptex}}; \quad  {\color{red}\textbf{  Pour préparer le #2}}
\input{#3}
}% fin condition2 vraie
{%condition2 fausse
\vspace{0.4cm}
\stepcounter{cptex}
\tikz\node[rounded corners=0pt,draw,fill=red]{\color{white}\textbf{ \thecptex}}; \quad  {\color{red}\textbf{  Pour préparer le #2}} En temps libre
}% fin condition2 fausse
} % fin de la procédure



\newcommand{\DNB}[3]{
\ifthenelse{\equal{#1}{1}}
{%condition2 vraie
\vspace{0.4cm}
\stepcounter{cptex}
\tikz\node[rounded corners=0pt,draw,fill=purple]{\color{white}\textbf{ \thecptex}}; \quad  {\color{purple}\textbf{  Vu au brevet #2}}
\input{#3}
}% fin condition2 vraie
{%condition2 fausse
\vspace{0.4cm}
\stepcounter{cptex}
\tikz\node[rounded corners=0pt,draw,fill=purple]{\color{white}\textbf{ \thecptex}}; \quad  {\color{purple}\textbf{  Vu au brevet #2}} En temps libre
}% fin condition2 fausse
} % fin de la procédure


\newcommand{\DNBS}[1]{ %DNB pour un séance sans input
\vspace{0.4cm}
\stepcounter{cptex}
\tikz\node[rounded corners=0pt,draw,fill=purple]{\color{white}\textbf{ \thecptex}}; \quad  {\color{purple}\textbf{  Vu au brevet #1}}
}% fin condition2 vraie

\newcommand{\DNBC}[4]{
\ifthenelse{\equal{#1}{1}}
{%condition2 vraie
\vspace{0.4cm}
\stepcounter{cptex}
\tikz\node[rounded corners=0pt,draw,fill=purple]{\color{white}\textbf{ \thecptex}}; \quad  {\color{purple}\textbf{  Vu au brevet #2 }#4}
\input{#3}
}% fin condition2 vraie
{%condition2 fausse
\vspace{0.4cm}
\stepcounter{cptex}
\tikz\node[rounded corners=0pt,draw,fill=purple]{\color{white}\textbf{ \thecptex}}; \quad  {\color{purple}\textbf{  Vu au brevet #2} #4} En temps libre
}% fin condition2 fausse
} % fin de la procédure




\newcommand{\CR}[2]{
\ifthenelse{\equal{#1}{1}}
{%condition2 vraie
\vspace{0.4cm}
\stepcounter{cptex}
\tikz\node[rounded corners=0pt,draw,fill=red]{\color{white}\textbf{ \thecptex}}; \quad  {\color{red}\textbf{ Compte rendu}}
\input{#2}
}% fin condition2 vraie
{%condition2 fausse
\stepcounter{cptex}
\vspace{0.4cm}
\tikz\node[rounded corners=0pt,draw,fill=red]{\color{white}\textbf{ \thecptex}}; \quad  {\color{red}\textbf{  Compte rendu}} En temps libre
}% fin condition2 fausse
} % fin de la procédure


\newcommand{\avenir}{
\tikz\node[rounded corners=2pt,draw,fill=gray]{\color{white}   \textbf{ Parcours Avenir}};
} % fin de la procédure
\newcommand{\citoyen}{
\tikz\node[rounded corners=2pt,draw,fill=bleu1]{\color{white}  \textbf{ Parcours Citoyen}};
} % fin de la procédure
\newcommand{\sante}{
\tikz\node[rounded corners=2pt,draw,fill=green]{\color{white}   \textbf{ Parcours Santé}};
} % fin de la procédure
\newcommand{\PAEC}{
\tikz\node[rounded corners=2pt,draw,fill=purple]{\color{white}   \textbf{ Parcours d'éducation artistique et culturel}};
} % fin de la procédure





\newcommand{\Parcours}[3]{
\ifthenelse{\equal{#1}{1}}
{%condition2 vraie
\vspace{0.4cm}
\stepcounter{cptex}
\tikz\node[rounded corners=0pt,draw,fill=vert]{\color{white}\textbf{ \thecptex. Parcours #2}}; 
\input{#3}
}% fin condition2 vraie
{%condition2 fausse
\vspace{0.4cm}
\stepcounter{cptex}
\tikz\node[rounded corners=0pt,draw,fill=vert]{\color{white}\textbf{ \thecptex. Parcours #2}}; \quad En temps libre
}% fin condition2 fausse
} % fin de la procédure


\newcommand{\Neutre}[2]{
\ifthenelse{\equal{#1}{1}}
{%condition2 vraie
\input{#2}
}% fin condition2 vraie
{%condition2 fausse
En temps libre
}% fin condition2 fausse
} % fin de la procédure


\newcommand{\EL}[1]{%
\vspace{0.4cm}
\tikz\node[rounded corners=0pt,draw,fill=violet]{\color{white}\textbf{.. }; \quad  \href{#1}{\color{violet}\textbf{ En ligne}}
}
}

%%%%%%%%%%%%% Exercice corrigé
\newenvironment{cor}[2][]{%
\vspace{0.4cm}
\begin{bclogo}[couleur=gray!30, arrondi =0.15, noborder=true, couleurBarre=yellow, logo = \bcoutil ]{ 
\normalsize{Elements de correction #1 #2}}}
{%
\end{bclogo}
\par}

\newcommand{\ROC}{
\tikz\node[rounded corners=5pt,draw,fill=red]{\color{white}\textbf{ROC}}; \quad  {\color{red}\textbf{Restitution Organisée de Connaissance}}
}


\newcommand{\Rech}{
\vspace{0.4cm}
\stepcounter{cptex}
\tikz\node[rounded corners=0pt,draw,fill=bleu3]{\color{white}\textbf{ \thecptex}}; 
\tikz\node[rounded corners=0pt,draw,fill=bleu3]{\color{white}\textbf{ Situation de recherche}}; 
}


\newcommand{\SR}{
\vspace{0.4cm}
\stepcounter{cptex}
\tikz\node[rounded corners=0pt,draw,fill=bleu2]{\color{white}\textbf{ Situation de recherche \thecptex}}; 
\\
}


%%%%%%%%%%%%% enonce
\newenvironment{enonce}[2][]{%
\begin{tcolorbox}[breakable, enhanced,widget, colback=white!10!white,boxrule=0pt,frame hidden,
borderline west={2mm}{0mm}{#1}]
\textbf{ {\color{#1} ÉNONCÉ }} #2}
{%
\end{tcolorbox}
\par}
%%%%%%%%%%%%% Preuve
\newenvironment{cadre}[2][]{%
\begin{tcolorbox}[breakable, enhanced,widget, colback=#1!10!white,boxrule=0pt,frame hidden,
borderline west={2mm}{0mm}{#1}]
 #2}
{%
\end{tcolorbox}
\par}


\newcommand{\ExeU}{
\vspace{0.4cm}
\tikz\node[rounded corners=0pt,draw,fill=eduscol4B!40]{\color{white}\textbf{ Exercice de niveau 1}}; 
\\
}

\newcommand{\ExeD}{
\vspace{0.4cm}
\tikz\node[rounded corners=0pt,draw,fill=eduscol4B!60]{\color{white}\textbf{ Exercice de niveau 2}}; 
\\
}

\newcommand{\ExeT}{
\vspace{0.4cm}
\tikz\node[rounded corners=0pt,draw,fill=eduscol4B!80]{\color{white}\textbf{ Exercice de niveau 3}}; 
\\
}

\newcommand{\ExeQ}{
\vspace{0.4cm}
\tikz\node[rounded corners=0pt,draw,fill=eduscol4B]{\color{white}\textbf{ Exercice de niveau 4}}; 
\\
}

\newcommand{\CompU}{
\vspace{0.4cm}
\tikz\node[rounded corners=0pt,draw,fill=eduscol4P!40]{\color{white}\textbf{ Exercice de niveau 1}}; 
\\
}

\newcommand{\CompD}{
\vspace{0.4cm}
\tikz\node[rounded corners=0pt,draw,fill=eduscol4P!60]{\color{white}\textbf{ Exercice de niveau 2}}; 
\\
}

\newcommand{\CompT}{
\vspace{0.4cm}
\tikz\node[rounded corners=0pt,draw,fill=eduscol4P!80]{\color{white}\textbf{ Exercice de niveau 3}}; 
\\
}

\newcommand{\CompQ}{
\vspace{0.4cm}
\tikz\node[rounded corners=0pt,draw,fill=eduscol4P]{\color{white}\textbf{ Exercice de niveau 4}}; 
\\
}


\newcommand{\ExeComp}[1]{
\vspace{0.4cm}
\stepcounter{cptex}
\tikz\node[rounded corners=0pt,draw,fill=bleu2]{\color{white}\textbf{ \thecptex}}; \quad  {\color{bleu2}\textbf{#1}}
}% 



%%%%%%%%%%%%% Enigme
\newenvironment{Eni}[2][]{%
\medskip \begin{tcolorbox}[widget,colback=white!15,colframe=eduscol4P!75!white,
adjusted title= \stepcounter{cptdef} Énigme \thecptdef . {#1} \textit{#2}]}
{%
\end{tcolorbox}\par}


\newcommand{\sacado}[1]{
\vspace{0.4cm}
\stepcounter{cptex}
\tikz\node[rounded corners=0pt,draw,fill=vertS]{\color{white}\textbf{\thecptex . SACADO}}; \quad  {\color{vertS} Code exercice : \textbf{#1}}
}%